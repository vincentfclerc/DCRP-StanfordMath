\documentclass{article}
\usepackage{graphicx} % Required for inserting images
\usepackage[utf8]{inputenc}
\usepackage[T1]{fontenc}
\usepackage{amsmath,amssymb,amsfonts,amsthm}
\usepackage{lmodern}
\usepackage{setspace}

\title{questions-stanfordMath}
\author{vincent.clerc.ucbl }
\date{December 2024}

\begin{document}
\section{}
A body starts from a point with an initial speed of \( v_0 = 10 \, \mathrm{m/s} \) and a constant acceleration of \( a = 2 \, \mathrm{m/s^2} \). What is the distance traveled after \( t = 5 \, \mathrm{s} \)?

The distance traveled is given by the formula: \[ x = v_0 t + \frac{1}{2} a t^2 \] Substituting the values: \[ x = 10 \cdot 5 + \frac{1}{2} \cdot 2 \cdot 5^2 = 50 + 25 = 75 \, \mathrm{m} \] The distance traveled is therefore \( x = 75 \, \mathrm{m} \).

\section{}
A person walks at a constant speed of \( v = 1.5 \, \mathrm{m/s} \) for \( t = 10 \, \mathrm{s} \). What distance do they cover?

The distance traveled is given by: \[ x = v \cdot t \] Substituting the values: \[ x = 1.5 \cdot 10 = 15 \, \mathrm{m} \] The distance traveled is therefore \( x = 15 \, \mathrm{m} \).

\section{}
An object falls freely from rest. What distance does it cover after \( t = 3 \, \mathrm{s} \), assuming a gravitational acceleration of \( g = 9.8 \, \mathrm{m/s^2} \)?

The distance traveled is given by: \[ x = \frac{1}{2} g t^2 \] Substituting the values: \[ x = \frac{1}{2} \cdot 9.8 \cdot 3^2 = 4.9 \cdot 9 = 44.1 \, \mathrm{m} \] The distance traveled is therefore \( x = 44.1 \, \mathrm{m} \).

\section{}
A projectile is launched vertically upward with an initial velocity of \( v_0 = 20 \, \mathrm{m/s} \). What is the maximum height reached by the projectile? Assume \( g = 9.8 \, \mathrm{m/s^2} \).

The maximum height is given by: \[ h = \frac{v_0^2}{2 g} \] Substituting the values: \[ h = \frac{20^2}{2 \cdot 9.8} = \frac{400}{19.6} _x0007_\approx 20.41 \, \mathrm{m} \] The maximum height reached is therefore \( h _x0007_\approx 20.41 \, \mathrm{m} \).

\section{}
An electron is in a one-dimensional infinite potential well of width \( L = 1 \, \mathrm{nm} \). What is the wavelength associated with the ground state (\( n = 1 \))?

For an infinite potential well, the allowed wavelengths are given by: \[ \lambda_n = \frac{2L}{n} \] For \( n = 1 \), the wavelength is: \[ \lambda_1 = \frac{2 \cdot 1 \, \mathrm{nm}}{1} = 2 \, \mathrm{nm} \] In strict SI units: \[ \lambda_1 = 2 \cdot 10^{-9} \, \mathrm{m} \] The associated wavelength is therefore \( \lambda_1 = 2 \, \mathrm{nm} \) or \( 2 \cdot 10^{-9} \, \mathrm{m} \).

\section{}
Consider a dense star (such as a white dwarf) with a radius of \( R = 5000 \, \mathrm{km} \). What is the Schwarzschild radius (\( R_s \)) associated if the star collapses into a black hole? Use \( G = 6.674 \cdot 10^{-11} \, \mathrm{m^3 \cdot kg^{-1} \cdot s^{-2}} \), \( c = 3.00 \cdot 10^8 \, \mathrm{m/s} \), and a typical stellar mass \( M = 2 \cdot 10^{30} \, \mathrm{kg} \).

The Schwarzschild radius is given by: \[ R_s = \frac{2 G M}{c^2} \] Substituting the values: \[ R_s = \frac{2 \cdot 6.674 \cdot 10^{-11} \cdot 2 \cdot 10^{30}}{(3.00 \cdot 10^8)^2} \] \[ R_s = \frac{2.6696 \cdot 10^{20}}{9 \cdot 10^{16}} _x0007_\approx 2.966 \cdot 10^3 \, \mathrm{m} \] The Schwarzschild radius is therefore \( R_s _x0007_\approx 2.97 \, \mathrm{km} \) or \( 2.97 \cdot 10^3 \, \mathrm{m} \).

\section{}
An ultra-relativistic neutrino travels with an energy of \( E = 1 \, \mathrm{TeV} \). What is its associated Compton wavelength? (Use \( h = 6.626 \cdot 10^{-34} \, \mathrm{J \cdot s} \) and \( c = 3.00 \cdot 10^8 \, \mathrm{m/s} \)).

The wavelength is given by: \[ \lambda = \frac{h c}{E} \] Converting \( E = 1 \, \mathrm{TeV} \) to joules (\( 1 \, \mathrm{TeV} = 1.602 \cdot 10^{-7} \, \mathrm{J} \)): \[ E = 1.602 \cdot 10^{-7} \, \mathrm{J} \] Substituting: \[ \lambda = \frac{6.626 \cdot 10^{-34} \cdot 3.00 \cdot 10^8}{1.602 \cdot 10^{-7}} \] \[ \lambda _x0007_\approx 1.24 \cdot 10^{-18} \, \mathrm{m} \] The wavelength is therefore \( \lambda _x0007_\approx 1.24 \cdot 10^{-18} \, \mathrm{m} \).

\section{}
A car travels at a constant speed of \( v = 60 \, \mathrm{mph} \) for \( t = 2 \, \mathrm{hours} \). How far does it travel?

Distance is given by \[ x = v \cdot t \]. Substituting the values: \[ x = 60 \, \mathrm{mph} \times 2 \, \mathrm{h} = 120 \, \mathrm{miles} \]. The distance traveled is therefore \( x = 120 \, \mathrm{miles} \).

\section{}
A mass-spring system has \( k = 200 \, \mathrm{N/m} \) and \( m = 0.5 \, \mathrm{kg} \). What is the angular frequency \(\omega\)?

The angular frequency for a mass-spring system is \(\omega = \sqrt{\frac{k}{m}}\). Substituting: \(\omega = \sqrt{\frac{200}{0.5}} = \sqrt{400} = 20 \, \mathrm{rad/s}\).

\section{}
A marathon length is a\approximately \(26.2 \, \mathrm{miles}\). Convert this distance into kilometers, given \(1 \, \mathrm{mile} _x0007_\approx 1.609 \, \mathrm{km}\).

Converting: \[ 26.2 \, \mathrm{miles} \times 1.609 \, \mathrm{km/mile} _x0007_\approx 42.16 \, \mathrm{km} \]. The distance is therefore \( 42.16 \, \mathrm{km} \).

\section{}
A star with luminosity \( L = 3.828 \times 10^{26} \, \mathrm{W} \) (like the Sun) illuminates a planet at distance \( d = 1.496 \times 10^{11} \, \mathrm{m} \). What is the intensity at that distance?

Intensity is given by \[ I = \frac{L}{4 \pi d^2} \]. Substituting: \( d^2 = (1.496 \times 10^{11})^2 _x0007_\approx 2.238 \times 10^{22} \), and \(4\pi _x0007_\approx 12.566\). Thus, denominator \(_x0007_\approx 12.566 \times 2.238 \times 10^{22} _x0007_\approx 2.81 \times 10^{23}\). \[ I = \frac{3.828 \times 10^{26}}{2.81 \times 10^{23}} _x0007_\approx 1.36 \times 10^3 \, \mathrm{W/m^2} \]. A\approximately \( I _x0007_\approx 1360 \, \mathrm{W/m^2} \).

\section{}
A cyclist rides at a speed of \( v = 20 \, \mathrm{km/h} \) for \( t = 30 \, \mathrm{minutes} \). What distance do they cover in meters?

First convert time: \(30 \, \mathrm{min} = 0.5 \, \mathrm{h}\). Distance: \( x = v \cdot t = 20 \, \mathrm{km/h} \times 0.5 \, \mathrm{h} = 10 \, \mathrm{km} \). Converting to meters: \(10 \, \mathrm{km} = 10{,}000 \, \mathrm{m}\).

\section{}
A comet travels at \( v = 40 \, \mathrm{km/s} \) for \( t = 1 \, \mathrm{day} \). How far does it travel in kilometers? (1 day = 86400 s)

Calculate: \( x = v \cdot t = 40 \, \mathrm{km/s} \times 86400 \, \mathrm{s} = 40 \times 86400 = 3{,}456{,}000 \, \mathrm{km} \).

\section{}
An object moves at \( 50 \, \mathrm{cm/s} \) for \( 200 \, \mathrm{s} \). How far does it travel in meters?

Distance: \( x = v \cdot t = 50 \, \mathrm{cm/s} \times 200 \, \mathrm{s} = 10{,}000 \, \mathrm{cm} \). Converting to meters: \(10{,}000 \, \mathrm{cm} = 100 \, \mathrm{m}\).

\section{}
A plane flies at \( v = 900 \, \mathrm{km/h} \) for \( t = 2.5 \, \mathrm{h} \). How far does it travel in kilometers?

Distance: \( x = v \cdot t = 900 \, \mathrm{km/h} \times 2.5 \, \mathrm{h} = 2250 \, \mathrm{km} \).

\section{}
A student runs at a constant speed of \( v = 2 \, \mathrm{m/s} \) for \( t = 300 \, \mathrm{s} \). How far does the student run in meters?

Distance: \( x = v \cdot t = 2 \times 300 = 600 \, \mathrm{m} \).

\section{}
A cyclist expends energy at a rate of \( 200 \, \mathrm{W} \) for \( 600 \, \mathrm{s} \). How much energy is used in Joules?

Energy: \( E = P \cdot t = 200 \times 600 = 120{,}000 \, \mathrm{J} \).

\section{}
A temperature changes from \( 20 \, ^\circ\mathrm{C} \) to \( 68 \, ^\circ\mathrm{F} \). Convert \( 68 \, ^\circ\mathrm{F} \) to Celsius using \( T_{C} = (T_{F}-32) \times \frac{5}{9} \).

Substitute: \( T_{C} = (68-32) \times \frac{5}{9} = 36 \times \frac{5}{9} = 20 \, ^\circ\mathrm{C} \).

\section{}
A truck travels \( 360 \, \mathrm{mi} \) at an average speed of \( 60 \, \mathrm{mph} \). How many hours does the journey take?

Time: \( t = \frac{360}{60} = 6 \, \mathrm{h} \).

\section{}
You deposit \$500 into a savings account with an annual interest rate of 5%. How much interest do you earn in one year (no compounding)?

Interest = Principal \( \times\) Rate = 500 \( \times\) 0.05 = \$25.

\section{}
A patient weighs \( 70 \, \mathrm{kg} \). A medication dose is \(2 \, \mathrm{mg/kg}\). How many milligrams of medication are administered?

Dose: \( 70 \times 2 = 140 \, \mathrm{mg} \).

\section{}
A car travels \( 100 \, \mathrm{km} \) at a speed of \( 50 \, \mathrm{km/h} \). How long does it take in hours?

Time: \( t = \frac{x}{v} = \frac{100}{50} = 2 \, \mathrm{h} \).

\section{}
An airplane flies at \( v = 250 \, \mathrm{m/s} \) for \( t = 10 \, \mathrm{min} \). What distance does it cover in kilometers? (1 min = 60 s)

Convert 10 min to 600 s: \( x = 250 \times 600 = 150{,}000 \, \mathrm{m} = 150 \, \mathrm{km} \).

\section{}
A force of \( F = 10 \, \mathrm{N} \) is applied to a mass of \( m = 2 \, \mathrm{kg} \). What is the acceleration in \(\mathrm{m/s^2}\)?

Acceleration: \( a = \frac{F}{m} = \frac{10}{2} = 5 \, \mathrm{m/s^2} \).

\section{}
A star is observed at a distance of \( 10 \, \mathrm{parsec} \). Convert this distance into light-years, given \(1 \, \mathrm{pc} _x0007_\approx 3.26 \, \mathrm{ly}\).

Distance in ly: \( 10 \times 3.26 = 32.6 \, \mathrm{ly} \).

\section{}
A data storage device has a capacity of \( 2 \, \mathrm{GB} \). Convert this to MB if \(1 \, \mathrm{GB} = 1024 \, \mathrm{MB}\).

\( 2 \, \mathrm{GB} = 2 \times 1024 = 2048 \, \mathrm{MB} \).

\section{}
A balloon of circumference is measured as \( 30 \, \mathrm{cm} \). If its diameter is a\approximately circumference/\(\pi\), find the diameter in cm (use \(\pi _x0007_\approx 3.14\)).

Diameter: \( d = \frac{30}{3.14} _x0007_\approx 9.55 \, \mathrm{cm} \).

\section{}
A spacecraft travels at a speed of \( v = 15 \, \mathrm{km/s} \) for \( t = 2 \, \mathrm{hours} \). What distance does it cover?

The distance is given by: \[ x = v \cdot t \] Substituting: \[ x = 15 \cdot 7200 = 108,000 \, \mathrm{km} \].

\section{}
A gas expands isothermally, doubling its volume. If the initial pressure was \( P_1 = 100 \, \mathrm{kPa} \), what is the final pressure?

For isothermal processes: \( P_1 V_1 = P_2 V_2 \). Since volume doubles, \( P_2 = \frac{P_1}{2} = 50 \, \mathrm{kPa} \).

\section{}
An ideal gas undergoes adiabatic expansion. If the initial temperature is \( T_1 = 300 \, \mathrm{K} \) and the final temperature is \( T_2 = 150 \, \mathrm{K} \), what is the ratio of the final volume to the initial volume (\( V_2 / V_1 \))?

For an adiabatic process: \( T_1 V_1^{\gamma-1} = T_2 V_2^{\gamma-1} \). Assuming air (\( \gamma = 1.4 \)), the ratio is: \( \left(\frac{T_1}{T_2} \right)^{1/(\gamma-1)} = 2.64 \).

\section{}
A black hole emits Hawking radiation at a temperature of \( T = 10^{-8} \, \mathrm{K} \). What is the corresponding peak wavelength? Use Wien's Law: \( \lambda = \frac{b}{T} \) where \( b = 2.9 \cdot 10^{-3} \, \mathrm{m \cdot K} \).

Using Wien's Law: \( \lambda = \frac{2.9 \cdot 10^{-3}}{10^{-8}} = 2.9 \cdot 10^{5} \, \mathrm{m} \).

\section{}
A piston-cylinder device contains \( 0.1 \, \mathrm{m^3} \) of air at \( 300 \, \mathrm{K} \). The volume triples isobarically. What is the final temperature?

Using \( V_1 / T_1 = V_2 / T_2 \): \( T_2 = 3 \cdot 300 = 900 \, \mathrm{K} \).

\section{}
A chemical reaction releases \( Q = 5000 \, \mathrm{J} \) of heat. If this heat raises the temperature of \( m = 2 \, \mathrm{kg} \) of water, what is the temperature change? Assume specific heat capacity \( c = 4184 \, \mathrm{J/(kg \cdot K)} \).

Using \( Q = m c \Delta T \): \( \Delta T = \frac{5000}{2 \cdot 4184} = 0.598 \, \mathrm{K} \).

\section{}
A satellite orbits Earth at a distance of \( r = 6.7 \cdot 10^6 \, \mathrm{m} \). What is its orbital speed? Use \( g = 9.8 \, \mathrm{m/s^2} \).

Orbital speed is given by: \( v = \sqrt{g r} = \sqrt{9.8 \cdot 6.7 \cdot 10^6} = 8.1 \cdot 10^3 \, \mathrm{m/s} \).

\section{}
A fluid flows through a pipe with a cross-sectional area of \( A = 0.05 \, \mathrm{m^2} \) and velocity \( v = 2 \, \mathrm{m/s} \). What is the volume flow rate?

The volume flow rate is: \( Q = A v = 0.05 \cdot 2 = 0.1 \, \mathrm{m^3/s} \).

\section{}
An ideal Carnot engine operates between a hot reservoir at \( T_H = 600 \, \mathrm{K} \) and a cold reservoir at \( T_C = 300 \, \mathrm{K} \). The engine absorbs \( Q_H = 5000 \, \mathrm{J} \) of heat from the hot reservoir. What is the efficiency of the engine and how much work is done?

The efficiency of a Carnot engine is given by: \[ \eta = 1 - \frac{T_C}{T_H} \]. Substituting: \[ \eta = 1 - \frac{300}{600} = 0.5 \]. Work done: \[ W = \eta \cdot Q_H = 0.5 \cdot 5000 = 2500 \, \mathrm{J} \].

\section{}
A neutron star has a mass of \( M = 1.4 \, M_\odot \) and a radius of \( R = 10 \, \mathrm{km} \). What is the escape velocity at its surface?

Escape velocity is given by: \[ v_e = \sqrt{\frac{2 G M}{R}} \]. Substituting values, \[ v_e \approx 6.09 \cdot 10^8 \, \mathrm{m/s} \].

\section{}
Water flows through a horizontal pipe with a cross-sectional area of \( A_1 = 0.2 \, \mathrm{m^2} \) at a velocity of \( v_1 = 3 \, \mathrm{m/s} \). The pipe narrows to a cross-sectional area of \( A_2 = 0.05 \, \mathrm{m^2} \). What is the velocity of water in the narrower section?

Using the continuity equation: \[ A_1 v_1 = A_2 v_2 \]. Substituting: \[ 0.2 \cdot 3 = 0.05 \cdot v_2 \]. Solving: \[ v_2 = 12 \, \mathrm{m/s} \].

\section{}
A chemical reaction releases \( Q = 1500 \, \mathrm{J} \) of energy into \( 0.5 \, \mathrm{kg} \) of water. If the specific heat capacity of water is \( c = 4184 \, \mathrm{J/(kg \cdot K)} \), what is the temperature change?

Using the heat equation: \[ Q = m c \Delta T \]. Rearranging: \[ \Delta T = \frac{Q}{m c} \]. Substituting: \[ \Delta T = \frac{1500}{0.5 \cdot 4184} = 0.717 \, \mathrm{K} \].

\section{}
A satellite orbits Earth at a distance of \( r = 7.0 \cdot 10^6 \, \mathrm{m} \). Using Earth's gravitational constant \( g = 9.8 \, \mathrm{m/s^2} \), calculate its orbital speed.

Orbital speed is given by: \[ v = \sqrt{g r} \]. Substituting: \[ v = \sqrt{9.8 \cdot 7.0 \cdot 10^6} = 8.29 \cdot 10^3 \, \mathrm{m/s} \].

\section{}
A photon has an energy of \( E = 4.0 \, \mathrm{eV} \). Convert this energy into joules using \( 1 \, \mathrm{eV} = 1.602 \cdot 10^{-19} \, \mathrm{J} \).

Energy in joules: \[ E = 4.0 \cdot 1.602 \cdot 10^{-19} = 6.408 \cdot 10^{-19} \, \mathrm{J} \].

\section{}
A mass-spring-damper system is described by the differential equation: \[ m \frac{d^2 x}{dt^2} + c \frac{dx}{dt} + kx = 0 \], where \( m = 1 \, \mathrm{kg} \), \( c = 2 \, \mathrm{Ns/m} \), and \( k = 5 \, \mathrm{N/m} \). Determine the natural frequency of the system and classify the damping.

The characteristic equation is given by: \[ m r^2 + c r + k = 0 \]. Substituting the values: \[ 1 r^2 + 2 r + 5 = 0 \]. Solving: \[ r = \frac{-2 \pm \sqrt{2^2 - 4(1)(5)}}{2(1)} \]. Simplifying: \[ r = \frac{-2 \pm \sqrt{-16}}{2} \]. \[ r = -1 \pm 2i \]. The system is underdamped with a natural frequency: \[ \omega_n = \sqrt{\frac{k}{m}} = \sqrt{5} = 2.236 \, \mathrm{rad/s} \].

\section{}
A medication dosage is calculated based on a patient's weight of \( 80 \, \mathrm{kg} \). The prescribed dosage is \( 2.5 \, \mathrm{mg/kg} \). What is the total medication dose required?

The total dosage is given by: \[ D = 80 \cdot 2.5 = 200 \, \mathrm{mg} \].

\section{}
In a financial account, an initial deposit of $1000 earns an annual interest rate of 5% compounded annually. How much money is in the account after 3 years?

Using compound interest: \[ A = P (1 + r)^t \]. Substituting: \[ A = 1000 (1 + 0.05)^3 = 1157.63 \, \mathrm{USD} \].

\section{}
An electron is confined in a one-dimensional quantum well of width \( L = 2 \, \mathrm{nm} \). What is the energy of the first excited state? Use the formula: \[ E_n = \frac{n^2 h^2}{8 m_e L^2} \], where \( h = 6.626 \cdot 10^{-34} \, \mathrm{J \cdot s} \), \( m_e = 9.109 \cdot 10^{-31} \, \mathrm{kg} \), and \( n = 2 \).

Substituting into the energy formula: \[ E_2 = \frac{2^2 (6.626 \cdot 10^{-34})^2}{8 (9.109 \cdot 10^{-31}) (2 \cdot 10^{-9})^2} \]. Solving: \[ E_2 \approx 1.50 \cdot 10^{-19} \, \mathrm{J} \].

\section{}
A train starts from rest and accelerates uniformly at \( a = 2 \, \mathrm{m/s^2} \) for \( t = 10 \, \mathrm{s} \). What distance does it cover in that time?

Using the formula: \[ x = \frac{1}{2} a t^2 \]. Substituting: \[ x = \frac{1}{2} (2) (10^2) = 100 \, \mathrm{m} \].

\section{}
A gas undergoes an isothermal expansion, doubling its initial volume. The initial pressure is \( P_1 = 100 \, \mathrm{kPa} \). What is the final pressure?

For isothermal expansion: \[ P_1 V_1 = P_2 V_2 \]. Since volume doubles: \[ P_2 = \frac{P_1}{2} = 50 \, \mathrm{kPa} \].

\section{}
A block of mass \( m = 5 \, \mathrm{kg} \) is pushed with a force of \( F = 20 \, \mathrm{N} \) on a frictionless surface. What is its acceleration?

Using Newton's second law: \[ a = \frac{F}{m} \]. Substituting: \[ a = \frac{20}{5} = 4 \, \mathrm{m/s^2} \].

\section{}
A wave travels with a frequency of \( f = 500 \, \mathrm{Hz} \) and a wavelength of \( \lambda = 0.68 \, \mathrm{m} \). What is its speed?

Wave speed is given by: \[ v = f \lambda \]. Substituting: \[ v = 500 \cdot 0.68 = 340 \, \mathrm{m/s} \].

\section{}
Convert a speed of \( 72 \, \mathrm{km/h} \) into meters per second.

To convert: \[ v = \frac{72 \cdot 1000}{3600} = 20 \, \mathrm{m/s} \].

\end{document}